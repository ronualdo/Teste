%\documentclass[a4paper,12pt]{article}   % Seu arquivo fonte precisa conter
\documentclass[12pt,oneside]{book}  %{abnt}
%\usepackage[brazilian]{babel}       %hifeniza em portugues
\usepackage[english,brazil]{babel}  % idiomas permitidos
%\usepackage[a4paper,top=30mm,bottom=20mm,left=30mm,right=30mm]{geometry}
\usepackage[T1]{fontenc}            %trata caracteres acentuados
\usepackage{ae}                     %resolve efeito colateral de [T1] em PDF
%\usepackage[latin1]{inputenc}   %edição direta com acentos
\usepackage[utf8]{inputenc}
\usepackage{a4wide}             % correta formatação da página em A4
\usepackage{times}              % uma fonte true type
\usepackage{fancyvrb}           % códigos Verbatim
%\usepackage{float}              % nem lembro
\usepackage{setspace}            % espaçamento duplo
\usepackage{fancyhdr}           % tratamento de cabeçalho
%\usepackage{xspace}             % nem lembro! 
\usepackage{subfigure}          % Subfiguras
\usepackage{amsthm,amsfonts}	% Símbolos matemáticos
% \usepackage{amsfonts}           % fontes matemáticas
\usepackage{amssymb}            % Símbolos matemáticos
 \usepackage{amsmath}            % Símbolos matemático
%\usepackage{apalike}            % referência bibliografica
\usepackage{multirow}			 %
\usepackage{algorithm}			%
\usepackage{algpseudocode}
\usepackage{indentfirst}		%indentar o primeiro paragrafo

%\usepackage{subtable}
% enisando a correta separação de sílabas
\hyphenation{es-ta-be-le-ci-do a-de-qua-da-men-te pro-ble-mas
   di-men-sio-na-men-to mo-de-lo
   }
%\usepackage{graphicx}
\usepackage[pdftex]{color,graphicx}


\usepackage[hang]{footmisc} %nota de rodape
\setlength{\footnotemargin}{1em}
%=====================================================================
% Fazendo algumas definições
%=====================================================================
%\geometry{headheight=7mm,headsep=3mm,footskip=7mm}
%\setlength{\parskip}{2ex plus0.5ex minus0.5ex}

% Para ajustar o espaçamento entre linhas,
% basta alterar o valor de 'baselinestretch'

% Espaçamento simples
% Comentar todos os comandos 'renewcommand'

% Espaçamento 1/2 - descomente apenas um
%\renewcommand{\baselinestretch}{1.25}  % Fonte 10pt
%\renewcommand{\baselinestretch}{1.21}  % Fonte 11pt
%\renewcommand{\baselinestretch}{1.24}   % Fonte 12pt

% Espaçamento duplo  - descomente apenas um
%\renewcommand{\baselinestretch}{1.67}  % Fonte 10pt
%\renewcommand{\baselinestretch}{1.62}  % Fonte 11pt
%\renewcommand{\baselinestretch}{1.66}  % Fonte 12pt

\setcounter{secnumdepth}{3}

% Opções de estilo da bibliografia, desmarque apenas uma!
%---------------------------------------------------------------------
%\bibliographystyle{plain}  % estilo para labels em números
%\bibliographystyle{alpha}  % estilo para labels em iniciais
\bibliographystyle{apalike} % estilo utilizado pela SBC
%\bibliographystyle{acm}    % estilo utilizado pela ACM
%---------------------------------------------------------------------

%=====================================================================
% Dados sobre a Dissertação
%=====================================================================


%=====================================================================
% Dados sobre o Programa de Pós-Graduação
%=====================================================================

% Nome da Instutuição
\newcommand{\instituicao}
           {Faculdade Natalense para o Desenvolvimento do RN}

% Nome do Centro
\newcommand{\centro}
           {Centro de Ciências Exatas e da Terra}

% Nome do Departamento
\newcommand{\departamento}
           {Departamento de Informática e Matemática Aplicada}

% Nome do Programa
\newcommand{\programa}
           {Curso de Ciências da Computação}

% Nome do Coordenador do Programa
\newcommand{\coordenador}
           {Prof. Dr. Bruno Motta de Carvalho}

% Grau obtido pelo programa
\newcommand{\grau}
           {Bacharel em Ciências da Computação (bsc.)}

% Grau (em inglês) obtido pelo programa
\newcommand{\grauEMingles}
           {Master in Systems and Computation (MSc.)}


%=====================================================================
% Dados sobre a Banca Examinadora (não incluir orientador)
%=====================================================================

% Nome do Presidente
\newcommand{\presidente}
           {Prof. Dr. Bruno Motta de Carvalho }

% Nome do 1º Membro
\newcommand{\membroUM}
           {Profa. Dra.  Elizabeth Ferreira Gouvêa Goldbarg}


% Nome do 3º Membro
%\newcommand{\membroTRES}
%           {}


%=====================================================================
% Dados sobre a Dissertação
%=====================================================================

% Nome do Autor da Dissertação
\newcommand{\autor}
           {José Diego Saraiva da Silva}

% Nome do Autor (para Ficha Catolográfica) da Dissertação
\newcommand{\autorFicha}
           {Silva, José Diego Saraiva}

% Título da Dissertação
\newcommand{\titulo}
           {\textbf{Uma Plataforma para Sistemas Embarcados:} Desenvolvimento de um 
			Processador RTL usando SystemC}
\newcommand{\tipo}
			{RELATÓRIO DE GRADUAÇÃO}
% Título (em inglês) da Dissertação
%\newcommand{\tituloEMingles}
%           {}

% Título da Dissertação para Ficha Catalográfica (apenas primeria letra maiúscula)
\newcommand{\tituloFicha}
           {\textbf{Uma Plataforma para Sistemas Embarcados:}Desenvolvimento de um 
			processador RTL usando SystemC}

% Área da Dissertação
\newcommand{\area}
           {Sistemas Embarcados}

% Área (em inglês) da Dissertação
\newcommand{\areaEMingles}
           {Computational Mathematics}


% Palavras Chaves da Dissertação
\newcommand{\palavras}
           {Sistemas Embarcados; NoCS; SoCs
            SPARC V8; Processador; SystemC.}

% Palavras Chaves (em inglês) da Dissertação
\newcommand{\palavrasEMingles}
           {SPARC, System-on-Chip, Network-on-Chip,Platform-Based, Design, Cache.}

% Nome do Orientador
\newcommand{\orientador}
           {Prof. Dr. Ivan Saraiva Silva}

% Local
\newcommand{\local}
           {Natal/RN}

% Data
\newcommand{\data}
           {Junho de 2006}

% Data (em inglês)
\newcommand{\dataEMingles}
           {February, 2006}
% Ano para Ficha Catalográfica
\newcommand{\ano}
           {2006}
\newcommand{\tipoficha}
	   {Dissertação (Bacharel) }                 % Edite este arquivo com os seus dados
% \renewcommand{\baselinestretch}{1.0}   % Fonte 12pt
\begin{document}
%\renewcommand{\ALG@name}{Algoritmo}
%\newcommand{\algorithmicthen}{\textbf{entao}}
%\renewcommand{\listalgorithmname}{Lista de Algoritmos}
%=====================================================================
% Definições para as págias iniciais
%=====================================================================
\pagestyle{empty}
\pagenumbering{roman}

%=====================================================================
% Páginas iniciais 
%=====================================================================
% \include{frontpages/capa}
 \setcounter{page}{1} 
% \include{frontpages/folha-rosto}
\pagestyle{plain}

%%--- Só é utilizado na Dissertação ---%%
% \include{frontpages/resumo}
% \include{frontpages/ficha}
% \include{frontpages/dedicatoria}
% \include{frontpages/agradecimentos}
%
 \tableofcontents
 \listoffigures
 \listoftables
 %\include{lista_de_siglas}           % Lista de Siglas

\label{frontpages}

\clearpage

 \pagestyle{headings}

%=====================================================================
% Definição do formato de cabeçalho (package fancyhdr.sty)
% Este material deveria estar em um local mais apropriado
%=====================================================================

% Definir um estilo para páginas completas (cabeçalho + rodapé)
% \pagestyle{fancyplain}

% Marca de capítulo do tipo "2. Blábláblá..." no cabeçalho
%  \renewcommand{\chaptermark}[1]{\markboth{\thechapter.\ {#1}}{}}

% Para aumentar o tamanho da caixa para o cabeçalho
%  \addtolength{\headheight}{\baselineskip}

% Definindo o conteúdo do cabeçalho...
  \fancyhf{}
  \fancyhead[LO,LE]{\nouppercase{\textsf{\leftmark}}}
  \fancyhead[RO,RE]{\thepage}

% Para garantir que a primeira página de cada capítulo...
  \fancypagestyle{plain}{%
  \fancyhf{}%                         % não terá head ou foot
  \renewcommand{\headrulewidth}{0pt}} % nem linhas de cabeçalho

% Capítulos são numerados normalmente
\pagenumbering{arabic}
\setcounter{page}{1}

%=====================================================================
% Dissertação
%=====================================================================
\chapter{Introdução}

Uma propriedade intriseca dos softwares em um ambiente real é a sua necessidade
de evolução\cite{survey}. Apesar dessa necessidade ser uma caracteristica
própria, a manutenção e a evolução de software são caracterizadas pelo seu
grande custo e lenta velocidade de implementação\cite{maintenance}. Isso faz
necessário a busca de ferramentas e técnicas que permitam a redução da
complexidade através do aumento incremental da qualidade interna do
software\cite{survey}.

Muitos principios e práticas foram criados com o proposito de controlar a
duplicação. Considere, por exemplo, que todas as formas normais de banco de
dados de Codd servem para eliminar a duplicação de dados. Considere também como
a orientação orientada a objetos serve para concentrar o código classes bases
que de forma contrária seriam redundantes. Programação estruturada, programação
orientada a aspectos, programação orientada a componentes, são todas, em parte,
estratégias para elimininar a duplicação\cite{clean2009}.

Nesse trabalho iremos examinar o principio Don't Repeat Yourself. Em
\cite{97things2010} Steve Smith afirma que este este é talvez um dos mais
fundamentais principios da programação. Ele foi formulado por Andy Hunt e Dave
Thomas em ``The Pragmetic Programmer'', e permeia muitos outros padrões e
práticas do desenvolvimento de software\cite{97things2010}.

Durante este trabalho iremos apresentar o principio e alguns dos conceitos
relacionados, discutir a duplicação e por fim apresentar as conclusões geradas
pelo trabalho.

\section{Organização do Trabalho}
\chapter{DRY}

Em \cite{clean2009}, Robert C. Martin defende a importancia da programação
dizendo que apesar das ferramentas de geração de código, nós nunca estaremos
livres do código. Ainda segundo ele, o codigo representa o detalhamento dos
requisitos, sendo a linguagem utilizada para expressa-los.

Infelizmente, conhecimento não é estável, ele muda constantemente. Essa
instabilidade faz com que uma grande parte do tempo de desenvolvimento seja
gasto na manutenção do código, reorganizando e re-expressando o conhecimento do
sistema.

Ao realizar manutenções é necessário encontrar e mudar a representação do
conhecimento incorporado a aplicação. O problema é que conhecimento é fácil de
ser duplicado em especificações, processos e programas que nós
desenvolvemos\cite{pragmatic1999}.

Essa duplicação incha desnecessáriamente a base de código, resultando em mais
oportunidades para bugs e adicionando complexidade adicional para o sistema. O
inchaço que a duplicação tras ao sistema também o torna mais dificil de ser
totalmente compreendido pelos desenvolvedores do sistema, ou de ter certeza que
certas mudanças feitas em um lugar não precisam ser feitas em outras partes que
duplicam a lógica na qual se está trabalhando\cite{97things2010}.

Nesse contexto é que encontramos o principio DRY que diz que todo pedaço de
conhecimento deve ter uma única, universal e autoritativa representação dentro
de um sistemas. 

Quando seguido respeitando a estrutura, lógica, processo e função, o principio
DRY fornece um orientação fundamental para os desenvolvedores de software e
auxilia a criação de aplicações mais simples, manuteniveis e de alta qualidade..
\cite{97things2010}.

\section{Principios relacionados}

Existem uma série de outros príncipios de software relacionados ao príncipio
DRY. A seguir iremos apresentar alguns deles e falar sobre como esses principios
se relacionam.

\subsection{Principio Open/Closed}

Esse principio foi cunhado em 1988 por Bertran Meyer\cite{cgems}




%=====================================================================
% Incluindo bibliografia
%=====================================================================
%\nocite{*}         % listando todo o arquivo .bib
%\bibliography{referencias}
\bibliography{bibliografia/bibliografia}
 
%=====================================================================
% Dissertação estruturada em apdêndices, um arquivo para cada apêndice
%=====================================================================
%\appendix
%\include{apendice}     % Apendice exemplo
%\include{imml_resumo}      % Resumo da IMML
%\include{imml_translator}  % Especificação do tradutor
%\include{imml_sintaxe}     % Sintaxe da IMML
%\include{imml_esquema}     % Esquema da IMML
   
%---------------------------------------------------------------------
\label{final}          % referência para a última página
%---------------------------------------------------------------------
\end{document}