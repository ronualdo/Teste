\chapter{DRY}

Em \cite{clean2009}, Robert C. Martin defende a importancia da programação
dizendo que apesar das ferramentas de geração de código, nós nunca estaremos
livres do código. Ainda segundo ele, o codigo representa o detalhamento dos
requisitos, sendo a linguagem utilizada para expressa-los.

Infelizmente, conhecimento não é estável, ele muda constantemente. Essa
instabilidade faz com que uma grande parte do tempo de desenvolvimento seja
gasto na manutenção do código, reorganizando e re-expressando o conhecimento do
sistema.

Ao realizar manutenções é necessário encontrar e mudar a representação do
conhecimento incorporado a aplicação. O problema é que conhecimento é fácil de
ser duplicado em especificações, processos e programas que nós
desenvolvemos\cite{pragmatic1999}.

Essa duplicação incha desnecessáriamente a base de código, resultando em mais
oportunidades para bugs e adicionando complexidade adicional para o sistema. O
inchaço que a duplicação tras ao sistema também o torna mais dificil de ser
totalmente compreendido pelos desenvolvedores do sistema, ou de ter certeza que
certas mudanças feitas em um lugar não precisam ser feitas em outras partes que
duplicam a lógica na qual se está trabalhando\cite{97things2010}.

Nesse contexto é que encontramos o principio DRY que diz que todo pedaço de
conhecimento deve ter uma única, universal e autoritativa representação dentro
de um sistemas. 

Quando seguido respeitando a estrutura, lógica, processo e função, o principio
DRY fornece um orientação fundamental para os desenvolvedores de software e
auxilia a criação de aplicações mais simples, manuteniveis e de alta qualidade..
\cite{97things2010}.

\section{Principios relacionados}

Existem uma série de outros príncipios de software relacionados ao príncipio
DRY. A seguir iremos apresentar alguns deles e falar sobre como esses principios
se relacionam.

\subsection{Principio Open/Closed}

Esse principio foi cunhado em 1988 por Bertran Meyer\cite{cgems}


