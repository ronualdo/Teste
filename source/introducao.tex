\chapter{Introdução}

Uma propriedade intriseca dos softwares em um ambiente real é a sua necessidade
de evolução\cite{survey}. Apesar dessa necessidade ser uma caracteristica
própria, a manutenção e a evolução de software são caracterizadas pelo seu
grande custo e lenta velocidade de implementação\cite{maintenance}. Isso faz
necessário a busca de ferramentas e técnicas que permitam a redução da
complexidade através do aumento incremental da qualidade interna do
software\cite{survey}.

Muitos principios e práticas foram criados com o proposito de controlar a
duplicação. Considere, por exemplo, que todas as formas normais de banco de
dados de Codd servem para eliminar a duplicação de dados. Considere também como
a orientação orientada a objetos serve para concentrar o código classes bases
que de forma contrária seriam redundantes. Programação estruturada, programação
orientada a aspectos, programação orientada a componentes, são todas, em parte,
estratégias para elimininar a duplicação\cite{clean2009}.

Nesse trabalho iremos examinar o principio Don't Repeat Yourself. Em
\cite{97things2010} Steve Smith afirma que este este é talvez um dos mais
fundamentais principios da programação. Ele foi formulado por Andy Hunt e Dave
Thomas em ``The Pragmetic Programmer'', e permeia muitos outros padrões e
práticas do desenvolvimento de software\cite{97things2010}.

Durante este trabalho iremos apresentar o principio e alguns dos conceitos
relacionados, discutir a duplicação e por fim apresentar as conclusões geradas
pelo trabalho.

\section{Organização do Trabalho}